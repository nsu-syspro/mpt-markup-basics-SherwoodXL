\documentclass[12pt]{article}
\usepackage[utf8]{inputenc}
\usepackage[T2A]{fontenc}
\usepackage[russian]{babel}
\date{}
\title{Теорема Пифагора}
\begin{document}

\maketitle
В прямоугольном треугольнике квадрат гипотенузы равен сумме квадратов его катетов
\[ AB^2 = AC^2 + CB^2 \]
Чтобы получить значение обычной гипотенузы, нам нужно просто добавить корень над суммой квадратов
\[ AB = \sqrt{AC^2 + CB^2} \]

\textit{Пример:}
\[AC = 6; CB = 8 \Rightarrow AC^2 = 36; CB^2 = 64 \] 
\[AB = \sqrt{36 + 64} = \sqrt{100} = 10 \]

\end{document}