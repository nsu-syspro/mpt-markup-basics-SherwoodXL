\documentclass[12pt]{article}
\usepackage[utf8]{inputenc}
\usepackage[T2A]{fontenc}
\usepackage[russian]{babel}
\date{}
\title{Теорема Пифагора}
\begin{document}
\section{1\quadРешение квадратного уравнения}

\maketitle
\textit{Задача:} решить уравнение $2x^2 + 5x - 12 = 0$\\
\quad\textit{Решение.} Это квадратное уравнение, общий вид которого:
\[ax^2+bx+c=0\]
В нашем случае $a = 2, b = 5, c = -12$.\\
\quadСначала необходимо вычислить дискриминант уравнения:
\[D = b^2 - 4ac = (5)^2 - 4\cdot2\cdot(-12) = 121\]
Так как дискриминант является положительным $(D > 0)$, это уравнение имеет два корня, вычисляемые по формуле:

\[x{\tiny{1,2}}=\frac{-b\pm\sqrt{D}}{2a}=\frac{-5\pm11}{4}\]
Таким образом, $x{\tiny1}=\frac{-16}{4} = -4, x{\tiny2}=\frac{6}{4}=\frac{3}{2}$.\\
\quad\textit{Ответ:}$x{\tiny1}=-4, x{\tiny2}=\frac{3}{2}$.
\end{document}